\chapter{Concluzii}

\^{I}n lucrarea de fa\c{t}\u{a} s-a prezentat o metod\u{a} automat\u{a} prin care se poate calcula \^{i}n aproximativ 5-10 secunde frac\c{t}ia de ejec\c{t}ie la un pacient cu ajutorul unei re\c{t}ele neuronale de tip convolu\c{t}ional\u{a}, iar pe baza acestui calcul s\u{a} se stabileasc\u{a} dac\u{a} pacientul respectiv are sau nu probleme cu inima. Acest lucru se poate face \c{s}i de c\u{a}tre un medic specialist, \^{i}ns\u{a} timpul de calculare a frac\c{t}iei de ejec\c{t}ie este de aproximativ 20 de minute. Se poate trage concluzia c\u{a} prin aceast\u{a} metod\u{a} se reduce semnificativ timpul pe care un pacient trebuie s\u{a} \^{i}l a\c{s}tepte ca s\u{a} primesc\u{a} rezultatul de la medic \c{s}i \^{i}n plus medicul poate ac\c{t}iona mai repede pentru a ajuta pacien\c{t}ii bolnavi de inim\u{a}.

\par

Din p\u{a}cate aceast\u{a} metod\u{a} nu este tocmai perfect\u{a}, a\c{s}a cum am vazut mai sus, din aceast\u{a} cauz\u{a} unii pacien\c{t}i pot fi diagnostica\c{t}i ca fiind bolnavi dar \^{i}n realitate nu au nimic, sau invers s\u{a} fiu diagnostica\c{t}i ca fiind s\u{a}n\u{a}to\c{s}i dar de fapt ei au probleme cu inima. O mare problem\u{a} pe care o prezint\u{a} metoda abordat\u{a} \^{i}n lucrarea de licen\c{t}\u{a} este aceea c\u{a}, re\c{t}eaua neuronal\u{a} prezice de cele mai multe ori un ventricul st\^{a}ng mult prea mare dec\^{a}t este \^{i}n realitate, acest lucru dator\^{a}ndu-se faptului c\u{a} exist\u{a} \^{i}n exemple de antrenare mai multe ventricul st\^{a}ngi alfate la diastol\u{a} dec\^{a}t la sitol\u{a}, o solu\c{t}ie pentru acest lucru ar fi s\u{a} se multiple num\u{a}rul de imagini de antrenare c\^{a}nd inima se afl\u{a} la sistol\u{a}. O alt\u{a} solu\c{t}ie demn\u{a} de luat \^{i}n seama este aceea de a se folosii un alt tip de re\c{t}ea neuronal\u{a} pnetru detectarea ventricului st\^{a}ng, cea mai bun\u{a} arhitectura la ora actuala pentru acest lucru este cea dezvoltat\u{a} de c\u{a}tre Olaf Ronneberger, Philipp Fischer \c{s}i Thomas Brox de la Universitatea din Freiburg, Germania, botezat\u{a} U-net \cite{Unet}. S-a \^{i}ncercat implementarea unei arhitecturi similare la re\c{t}eaua neuronal\u{a} \^{i}ns\u{a} nu am ob\c{t}inut nici un rezultat iar implementarea ei este destul de complicat\u{a}.

\par

\^{I}n concluzie, metoda propus\u{a} \^{i}n lucrarea de fa\c{t}\u{a} este func\c{t}ional\u{a} \c{s}i \^{i}ntoarce rezultate de cele mai multe ori aproape de adev\u{a}r, fiind un punct de plecare destul de bun pentru implementarea unui sistem mult mai performant, care s\u{a} fie la fel de bun ca un medic specialist \^{i}ns\u{a} mult mai rapid \^{i}n problema stabilirii dac\u{a} un pacient sufer\u{a} sau nu cu sistemul cardiac. 