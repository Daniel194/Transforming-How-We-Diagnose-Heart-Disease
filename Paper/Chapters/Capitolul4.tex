\chapter{Identificarea bolilor cardiace cu re\c{t}ea neuronal\u{a}}

\^{I}nainte s\u{a} aplic\u{a}m o re\c{t}ea neuronal\u{a} peste setul nostru de date trebuie mai \^{i}nt\^{a}i s\u{a} le preproces\u{a}m pentru a le converti la un format mai adecvat, s\u{a} elimin\u{a}m unele date nedorite din setul de date care pot \^{i}ngreuna procesul de antrenare \c{s}i s\u{a} ad\u{a}ug\u{a}m unele date care ne vor fi folositoare pe parcurs.

\section{Preprocesarea datelor}

Deoarece datele furnizate de c\u{a}tre The National Heart, Lung, and Blood Institute sunt in format DICOM (Digital Imaging and Communications in Medicine), format destul de greu de lucrat, prima noastr\u{a} sarcin\u{a} va fi s\u{a} le convertim la un format mult mai propice, cum ar fi formatul PNG (Portable Network Graphics), pentru a putea cre\c{s}te viteza de calcul a re\c{t}elei neuronale. \^{I}nainte de a le convertii de la format DICOM la PNG putem s\u{a} mai extragem unele informa\c{t}ii de la fiecare radiografie pe care formatul DICOM le ofer\u{a}, cum ar fi spre exemplu  id-ul pacientului caruia \^{i}i apar\c{t}ine radiografia, numarul de linii \c{s}i de coloane a radiografiei, care definesc marimea , spa\c{t}iul \^{i}ntre pixeli, care reprezint\u{a} o pereche de numere care arat\u{a} distan\c{t}a fizic\u{a} dintre centrii a doi pixeli pe vertical\u{a}, respectiv orizontal\u{a}, acestea fiind m\u{a}surate in milimetrii, ad\^{a}ncimea radiografiei masurat\u{a} in milimetrii, pozi\c{t}ia imaginii, care reprezint\u{a} coordonatele pe axele x, y \c{s}i z, fa\c{t}\u{a} de col\c{t}ul de sus st\^{a}nga al imaginii a centrului  radiografiei facute, loca\c{t}ia relativ\u{a} a radiografiei reprezentat\u{a} in milimetrii \c{s}i axa de codificare a imaginii (dac\u{a} e pe coloan\u{a} sau pe r\^{a}nd ). Toate aceste informa\c{t}ii ne vor fi folositoare pe parcurs a\c{s}a c\u{a} le vom salva \^{i}ntr-un fi\c{s}ier CSV.

\par

Dup\u{a} citirea imaginii vom verifica dac\u{a} imaginea este orientat\u{a} pe coloan\u{a}, \^{i}n acest caz dac\u{a} este adev\u{a}rat  va trebuii s\u{a} calcul\u{a}m transpusa imaginii (liniile vor deveni coloane) \c{s}i s\u{a} o rotim pe orizontal\u{a} de-a lungul axei x, astfel \^{i}nc\^{a}t sus s\u{a} devin\u{a} jos \c{s}i invers, jos s\u{a} devin\u{a} sus, astfel c\u{a} la final imaginea va fi rotit\u{a} cu 90 de grade, iar dintr-o imagine orientat\u{a} pe coloan\u{a} vom avea o imagine orientat\u{a} pe r\^{a}nd. Imediat dup\u{a} aceea va trebuii s\u{a} redimension\u{a}m fiecare imagine, astfel \^{i}nc\^{a}t fiecare s\u{a} aib\u{a} 256 de pixeli \^{i}n \^{i}n\u{a}l\c{t}ime \c{s}i 256 de pixeli \^{i}n l\u{a}\c{t}ime, acest lucru se va face prin decuparea unui patrat de 256 x 256 de pixeli din imaginea original\u{a} care s\u{a} cuprind\u{a} doar inima pacientului, pentru a realiza acest lucru prima dat\u{a} vom verifica dac\u{a} imaginea are dimensiunile mai mici dec\^{a}t dorim noi s\u{a} decup\u{a}m, dac\u{a} da atunci va trebuii s\u{a} adaug\u{a}m un border negru \^{i}n jurul imaginii astfel \^{i}nc\^{a}t s\u{a} ating\u{a} dimensiunea dorit\u{a}. Dup\u{a} ce ne-am asigurat c\u{a} imaginea are o \^{i}n\u{a}l\c{t}ime \c{s}i o l\u{a}\c{t}ime mai mare dec\^{a}t vrem noi s\u{a} decup\u{a}m,  vom calcula punctele de start \c{s}i de final a decup\u{a}rii, astfel \^{i}nc\^{a}t s\u{a} lu\u{a}m fix centrul imaginii, acesta este un compromis destul de bun av\^{a}n \^{i}n vedere faptul c\u{a} fiecare radiografie are inima pacientului situat\u{a} in mijlocul ei.

\par

Acum c\u{a} avem o imagine de dimensiune 256x256 va trebuii s\u{a} aplic\u{a}m metoda CLAHE (Contrast Limited Adaptive Histogram Equalization) pentru a \^{i}mbun\u{a}t\u{a}\c{t}i contrastul \c{s}i calitatea imaginii. CLAHE este o variant\u{a} \^{i}mbunat\u{a}\c{t}it\u{a} a tehnicii de egalizare a histogramei (Histogram Equalization) a unei imagini gri, astfel \^{i}nc\^{a}t histograma aceasteia s\u{a} fie uniform\u{a}, iar fiecare valoare care poate fi \^{i}ntr-o imagine gri, s\u{a} aib\u{a} acela\c{s}i num\u{a}r aproximativ de pixeli. Astfel c\u{a}, fie o imagine f cu $m_r$ linii \c{s}i $m_c$ coloane \c{s}i cu pixeli care au valori \^{i}ntre 0 \c{s}i 255, vom nota cu p ca fiind probabilitatea ca un pixel s\u{a} aib\u{a} valorea n.

$$p_n = \frac{\text{numarul de pixeli cu valoarea n}}{\text{numarul total de pixeli}}$$

Unde n are valori cuprinse intre 0 \c{s}i 255. Atunci metoda de egalizare a histogramei pentru un pixel dintr-o imagine gri, la linia i \c{s}i la coloana c poate fi definit\u{a} \^{i}n felul urm\u{a}tor.

$$g_{i,j} = floor\bigg( 255 \sum_{n=0}^{f{i,j}} p_n \bigg)$$

\^{I}n formula de mai sus, floor face rotunjirea la cel mai apropiat num\u{a}r \^{i}ntreg inferior, iar g reprezint\u{a} noua imagine derivat\u{a} din f, care are histograma pixelilor egalizat\u{a}. Fa\c{t}\u{a} de tehnica de egalizare a histogramei care lucreaz\u{a} pe toat\u{a} imaginea, CLAHE aplic\u{a} acela\c{s}i principiu doar c\u{a} pe un bloc de pixeli, spre exemplu un bloc de pixeli de dimensiune 8x8, dintr-o imagine, acest lucru este necesar pentru a evita schimbarea contrastului \^{i}n regiuni unde acest lucru ar duce la deteliorarea calita\c{t}ii imaginii.

\begin{center}
\includegraphics[width=200]{before.png}
\includegraphics[width=200]{after.png}
\end{center}

Cele dou\u{a} imagini de mai sus reprezint\u{a} un exempu de imagine din setul de date \^{i}nainte de a fi procesat\u{a} \c{s}i dup\u{a} ce a fost procesat\u{a}, cum se poate observa imaginea procesat\u{a} a fost decupat\u{a} din mijlocul imaginii neprocesate, astfel \^{i}nc\^{a}t s\u{a} se elimine o cantitate c\^{a}t mai mare de date care nu sunt necesare pentru scopul nostru, de asemenea se mai poate observa c\u{a} imaginea procesat\u{a} are un contrast mai mare fa\c{t}\u{a} de imaginea neprocesat\u{a}, iar detaliile imaginii se pot observa mai bine, aceste lucruri vor duce la o vitez\u{a} de calcul \c{s}i la o acurate\c{t}e mai mare a re\c{t}elei neuronale.

\par

Cum am precizat mai sus, vom salva intr-un fi\c{s}ier CSV id-ul pacientului, numarul radiografiei, numarul imaginii, numarul de linii \c{s}i de coloane, distan\c{t}a fizic\u{a} \^{i}ntre centrii a doi pixeli, ad\^{a}ncimea radiografiei, locatia radiografiei, planul in care a fost facut\u{a} radiografia ( pe line sau pe coloan\u{a} ) \c{s}i pozi\c{t}ia imaginii fa\c{t}\u{a} de col\c{t}ul din st\^{a}nga sus, pe l\^{a}ng\u{a} toate acestea vom mai salva \c{s}i varianta prin care s-a facut radiografia, c\^{a}nd s-a f\u{a}cut radiografia, firma care a f\u{a}cut aparatul de RMN \c{s}i numele modelului, v\^{a}rsta pacientului, ziua de na\c{s}tere a pacientului, sexul pacientului, numele fi\c{s}ierului \c{s}i orientarea pacientului in imagine. DICOM define\c{s}te un sistem de coordonate numit RCS (Reference Coordinates System) prin care se stabile\c{s}te pozi\c{t}ia corpului  \^{i}n imagine, astfel c\u{a} direc\c{t}ia X este de la m\^{a}na dreapta a pacientului spre m\^{a}na st\^{a}ng\u{a} a acestuia, direc\c{t}ia Y este din fa\c{t}a pacientului spre spatele acestuia, iar direc\c{t}ia Z este de la picioare spre cap, din cauza faptului c\u{a} pozi\c{t}ia corpului este tridimensional\u{a}, iar radiografia este bidimensional\u{a} se va calcula proiec\c{t}ia fiecarei axe definite mai sus la axele unui plan bidimensional, \^{i}n acest caz \^{i}n formatul DICOM se vor gasi \c{s}ase valori care definesc orientarea pacientului in imagine. Primele trei valori reprezint\u{a} proiec\c{t}ia celor trei axe a planului tridimensional la axa X a planului bidimensional (Xx, Xy, Xz), iar celelalte trei valori reprezint\u{a} proiec\c{t}ia celor trei axe a planului tridimensional la axa Y a planului bidimensional (Yx, Yy, Yz), cu aceste valori se poate stabilii foarte u\c{s}or cum este orientat un pacient intr-o radiografie.

\par

Toate valorile salvate mai sus \^{i}n fi\c{s}ierul CSV au fost valori pe care nu am fost nevoi\c{t}i s\u{a} le calcul\u{a}m doarece sunt deja existente \^{i}n fiecare radiografie, \^{i}ns\u{a} pe baza lor putem s\u{a} calcul\u{a}m noi valori care ne vor fi de folos pe parcurs. Una dintre acestea este s\u{a} stabilim timpul dintre dou\u{a} imagini consecutive dintr-o radiografie \c{s}i distan\c{t}a dintre loca\c{t}iile lor, acestea pot fi u\c{s}or calculate prin diferen\c{t}a dintre timpii  celor dou\u{a} imagini \c{s}i a pozi\c{t}iilor pe care le au.

\section{Identificarea inimii}

Cum se poate observa \c{s}i \^{i}n imaginea de mai jos inima este destul de u\c{s}or de identificat cu ochiul liber pentru un om, \^{i}ns\u{a} pentru un calculator aceasta este \^{i}nc\u{a} greu de g\u{a}sit.

\par

Pentru a putea identifica inmia din astfel de imagine vom avea nevoie de o re\c{t}ea neuronal\u{a} care s\u{a} fac\u{a} segmentarea inimii de restul de date, iar dup\u{a} aceea inima va putea fi trecut\u{a} printr-o alt\u{a} re\c{t}ea neuronal\u{a} pentru a se calcula volum de s\^{a}nge care curge.

\par

Pentru antrenarea re\c{t}elei neuronale care va face segmentarea inimii vom avea nevoie de un alt set de date, deoarece setul de date pe care \^{i}l avem de la Institutul Na\c{t}ional pentru Inim\u{a}, Pl\u{a}m\^{a}ni \c{s}i S\^{a}nge din America nu ne ofer\u{a} date care s\u{a} indice pozi\c{t}ia exact\u{a} a inimii \^{i}n radiografie, \^{i}ns\u{a} spitalul Sunnybrook din Canada, Toronto pune la dispozi\c{t}ie date identice care ne ofer\u{a} pozi\c{t}ia exact\u{a} a inimii \^{i}n radiografie.

\begin{center}
\includegraphics[width=200]{after.png}
\end{center}

\subsection{Datele de la spitalul Sunnybrook}

Cum am precizat \c{s}i mai sus, spitalul Sunnybrook pune la dispozi\c{t}ie un set de date care arat\u{a} pozi\c{t}ia exact\u{a} a ventricului st\^{a}ng a inimii \^{i}n radiografie. Fiecare radiografie a fost f\u{a}cut\u{a} pe parcursul a unui ciclu a inimii, av\^{a}nd o dimensiune de [ 256 x 256 ] de pixeli, iar pentru fiecare radiografie este asociat  un fisier txt cu pozi\c{t}ia exact\u{a} a ventricului st\^{a}ng a inimii.

\begin{center}
\includegraphics[width=200]{sbh.png}
\includegraphics[width=200]{sbl.png}
\end{center}

\^{I}n imaginile de mai sus, avem o radiografie din setul de date Sunnybrook \^{i}n partea din st\^{a}nga, iar \^{i}n partea din dreapta avem pozi\c{t}ia exact\u{a} a ventricului st\^{a}ng \^{i}n radiografie, unde ce este colorat cu mov reprezint\u{a} faptul c\u{a} acolo nu se afl\u{a} ventriculul st\^{a}ng (culoarea mov are asociat\u{a} valoarea 0), iar ce este colorat cu galben reprezint\u{a} faptul c\u{a} acolo se afl\u{a} ventriculul st\^{a}ng (culoarea galben\u{a} are asociat\u{a} valoarea 1). Cu astfel de tipuri de date vom lucra pentru a antrena o re\c{t}ea neuronal\u{a} care s\u{a} fac\u{a} segmentarea ventricului st\^{a}ng a inimii de restul radiografiei. 

\par

La final dup\u{a} ce vom termina s\u{a} antren\u{a}m re\c{t}eaua neuronal\u{a}, vom introduce radiografii de la Institutul Na\c{t}ional pentru Inim\u{a}, Pl\u{a}m\^{a}ni \c{s}i S\^{a}nge din America, ca radiografia de mai sus din st\^{a}nga, iar la final vom ob\c{t}ine pozi\c{t}ia exact\u{a} a ventricului st\^{a}ng, ca \^{i}n imaginea din dreapta, unde 0 reprezint\u{a} faptul c\u{a} pixelul de la pozi\c{t}ia respectiv\u{a} nu apar\c{t}ine ventricului st\^{a}ng, iar valoarea 1 reprezint\u{a} faptul c\u{a} pixelul de la pozi\c{t}ia respectiv\u{a} apar\c{t}ine ventricului st\^{a}ng, iar pe baza acestor valori vom putea face segmentarea.

\subsection{Preprocesarea setului de date Sunnybrook}

Primul pas pe care trebuie s\u{a} \^{i}l facem \^{i}nainte de a introduce setul de date Sunnybrook \^{i}ntr-o re\c{t}ea neuronal\u{a} este de al preprocesa. La fel cum am facut \c{s}i mai sus, vom citi fiecare fi\c{s}ier DICOM, vom extrage din el doar radiografia, vom aplica metoda CLAHE pentru a \^{i}mbun\u{a}t\u{a}\c{t}ii contrastul \c{s}i calitatea radiografiei iar la final vom salva rezultatul ob\c{t}inut \^{i}ntr-un fi\c{s}ier PNG pentru a fi mai u\c{s}or de citit \c{s}i de vizualizat.

\subsection{Re\c{t}ele neuronale pentru segmentarea ventricului st\^{a}ng}

TODO