\chapter{Inroducere}

\section{Inima}

Inima este motorul organismului, organul care ne \c{t}ine \^{i}n via\c{t}\u{a}. Aceasta are rolul de a prelua s\^{a}ngele \^{i}nc\u{a}rcat cu oxigen \c{s}i substan\c{t}e nutritive venite de la plam\^{a}n \c{s}i de al pompa spre toate celelalte organe prin aorta si arterele care se desprind din ea. De asemenea tot inima este cea care prime\c{s}te s\^{a}ngele inc\u{a}rcat cu dioxid de carbon de la toate organele prin sistemul venos \c{s}i il pompeaz\u{a} spre pl\u{a}m\^{a}ni pentru oxigenare. Acest circuit se repet\u{a} pe tot parcursul vie\c{t}ii far\u{a} a se \^{i}ntrerupe , p\^{a}n\u{a} c\^{a}nd \^{i}nceteaz\u{a} s\u{a} mai bat\u{a} \c{s}i murim ...
\par 
Bolile de inim\u{a} sunt principala cauz\u{a} de deces la nivel mondial, acestea curm\^{a}nd via\c{t}a unui numar mare de persoane. Potrivit medicilor speciali\c{s}ti num\u{a}rul mare de decese cauzate de bolile cardiovasculare se afl\u{a} \^{i}n direct\u{a} legetur\u{a} cu dificultatea identific\u{a}rii simptomelor deoarece nu sunt \^{i}ntotdeauna evidente \c{s}i de cele mai multe ori sunt ignorate sau atribuite unei altei afec\c{t}iuni.
\par 
Din fericire, progresul tehnologic din ultimii ani ne-a adus posibilitatea de a crea noi metode mai eficiente de identificare a simptomelor bolilor cardiace, astfel \^{i}nc\^{a}t acestea s\u{a} pot fi tratate \^{i}nc\u{a} de la apari\c{t}ia lor. O astfel de metod\u{a}, care presupune identificarea bolilor cardiace folosind \^{i}nregistrarile RMN (Rezonan\c{t}\u{a} Magnetic\u{a} Nuclear\u{a}) a unei persoane, preprocesarea lor \c{s}i utilizarea unei re\c{t}ele neuronale pentru stabilirea faptului dac\u{a} persoana respectiv\u{a} prezint\u{a} simptome ale bolilor cardiace, va fi prezentat\u{a} \^{i}n paginile de mai jos. 

\section{Imaginea}

TODO

\subsection{Rezonan\c{t}\u{a} Magnetic\u{a} Nuclear\u{a}}

Rezonan\c{t}a Magnetic\u{a} Nuclear\u{a} se folose\c{s}te de un c\^{a}mp magnetic \c{s}i de unde de radiofrecven\c{t}\u{a} pentru vizualizarea diferitelor organe \c{s}i \c{t}esuturi ale corpului omenesc. Undele de radiofrecven\c{t}\u{a} sunt apoi traduse \^{i}ntr-o imagine.

\section{Re\c{t}elele neuronale}

Re\c{t}elele Neuronale au pornit de la ideea de a crea un model matematic care s\u{a} imite structura \c{s}i comportamentul unui creier uman.
\par
Creierul este compus din mai multe unit\u{a}\c{t}i numi\c{t}i neuroni, care comunic\u{a} \^{i}ntre ei prin sinapse, se aproximeaz\u{a} faptul c\u{a} creierul uman are aproximativ 86 de miliarde de neuroni \c{s}i 10^{14} - 10^{15}  sinapse.

\includegraphics[width=300]{neuron_small.png}

Fiecare neuron prime\c{s}te impulsuri prin dedridele sale de la al\c{t}i neuroni \c{s}i produce impulsuri prin axon pe care il transmite mai departe la al\c{t}i neuroni prin sinapse.

\par

Acest model biologic a incercat sa fie imitat de c\u{a}tre Warren McCulloch \c{s}i Walter Pitts \^{i}n anul 1943, astfle \^{i}ncat ace\c{s}tia au creat un model matematic care s\u{a} semene c\^{a}t mai mult cu varianta biologic\u{a}. \^{I}n modelul matematic propus de ace\c{s}tia datele de intrare primite prin dendrive ( s\u{a} le not\u{a}m cu \textbf{\textit{x}} ) sunt multiplicate cu ni\c{s}te \^{i}nt\u{a}riri ( s\u{a} le not\u{a}m cu \textbf{\textit{w}} ) pentru a se imita transferul facut prin sinapse \^{i}ntre axonul neuronului care transmite datle \c{s}i dendrivele neuronului care prime\c{s}te datele. Corpul neuronului a devenit un sumator care \^{i}nsumeaza produsul primit de la dendrive, astfel c\u{a} acesta se poate definii prin func\c{t}ia  $$ f = \sum_{i=1}^{n} x_i w_i $$ unde \textbf{\textit{n}} reprezinta numarul de dendrive. 